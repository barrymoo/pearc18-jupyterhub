\title{PEARC18 - Technical Paper Abstract}
\author{
    Barry Moore II$^a$ and Matt Burton$^b$\\
    Center for Research Computing$^a$ \\
    School of Computing and Information Science$^b$ \\
    University of Pittsburgh$^{a,b}$ \\
}
%\date{\today}

\documentclass[12pt]{article}

\usepackage{stix}
\usepackage{mathtools}
\usepackage{bm}
\usepackage{graphicx}
\usepackage{verbatim}
\usepackage{overcite}
\usepackage{xcolor}
\usepackage{hyperref}

\usepackage[letterpaper,ignoreall,top=2.65cm,bottom=2.7cm,%
left=2.0cm,right=2.0cm,foot=1cm,head=1cm]{geometry}

\newcommand{\mycite}[1]{Reference \citen{#1}}
\newcommand{\myfig}[1]{Figure \ref{#1}}
\newcommand{\myurl}[2]{\href{#1}{\color{blue}{#2}}}

\pagenumbering{gobble}

\begin{document}
\maketitle

\begin{abstract}
Jupyter notebooks are a platform for interactive computing and reproducible
research.  The notebooks are growing in popularity amongst computational
reseachers in traditional natural sciences as well as social sciences and
digital humanities. Notebooks blend code, prose, and data into a single ``computational
narrative'' that is sharable and publishable. JupyterHub is a system for managing
many user's Jupyter notebooks simultaneously. It provides plumbing for user authentication,
spawning notebook environments, and managing shared compute resources.

\begin{itemize}
    \item Authentication
    \begin{itemize}
        \item F5 \& Passport Setup (Kalpesh \& Anthony)
        \item Access Control
        \item Security and Institutional Identity Management
    \end{itemize}
    \item JupyterHub in Research Computing Environments
    \begin{itemize}
        \item Batch Computing
        \item Dealing with Heterogeneous Environments (Clusters)
    \end{itemize}
\end{itemize}

\begin{itemize}
    \item Future Directions/Considerations
    \begin{itemize}
        \item Temporary Notebooks on Compute Nodes (users without accounts)
        \item Sharing Notebooks in Classroom Environments
        \item User Configurable Notebook Environments and Reproducible Research w/ Singularity
    \end{itemize}
\end{itemize}

\end{abstract}

\end{document}
