\title{PEARC18 - Interplanetary Paradigm Shifts: Interactive Research Computing with Jupyter}
\author{
    Barry Moore II$^a$ and Matt Burton$^b$\\
    Center for Research Computing$^a$ \\
    School of Computing and Information Science$^b$ \\
    University of Pittsburgh$^{a,b}$ \\
}
%\date{\today}

\documentclass[12pt]{article}

\usepackage{stix}
\usepackage{mathtools}
\usepackage{bm}
\usepackage{graphicx}
\usepackage{verbatim}
\usepackage{overcite}
\usepackage{xcolor}
\usepackage{hyperref}

\usepackage[letterpaper,ignoreall,top=2.65cm,bottom=2.7cm,%
left=2.0cm,right=2.0cm,foot=1cm,head=1cm]{geometry}

 ewcommand{\mycite}[1]{Reference \citen{#1}}
\newcommand{\myfig}[1]{Figure \ref{#1}}
\newcommand{\myurl}[2]{\href{#1}{\color{blue}{#2}}}

\pagenumbering{gobble}

\begin{document}
\maketitle

\begin{abstract}
  Jupyter Notebooks are an increasingly popular platform for interactive
  computing and publishing reproducible research. They are being used for
  research and teaching across many disciplines in the natural sciences,
  computational social sciences, and digital humanities. Notebooks blend code,
  prose, and data into a single"computational narrative" making research
  workflows publishable, sharable and reproducible. JupyterHub is a system for
  managing multi-user Jupyter Notebooks environments, it provides infrastructure
  for user authentication, launching individual notebook servers, and managing
  shared compute resources. The benefits of Jupyter are a lower barrier of entry
  for new and non-traditional users of research computing systems, rapid
  feedback and results through interactive computation, and easy sharing of
  research workflows and data-rich results.

  Jupyter Notebooks and JupyterHub present a new paradigm for research
  computing, one that is interactive, web-based, and often at odds with the
  systems and interfaces for traditional batch computing (SLURM jobs and the SSH
  command line). In this technical paper, we will discuss how to configure
  Jupyter and JupyterHub to utilize existing batch computing resources, while
  allowing users, especially non-traditional users and students, to easily use
  Jupyter as an interactive computing environment. Optimal deployment of Jupyter
  and JupyterHub involves integration with institutional identity management
  systems, to leverage existing user management systems and security controls.
  Additionally, JupyterHub is used as a front-end to multiple computing clusters
  and connected to an existing batch queuing system (SLURM).  Additional
  features and capabilities of the Jupyter Platform will also be discussed.

\begin{itemize}
    \item Authentication
    \begin{itemize}
        \item F5 \& Passport Setup (Kalpesh \& Anthony)
        \item Access Control
        \item Security and Institutional Identity Management
    \end{itemize}
    \item JupyterHub in Research Computing Environments
    \begin{itemize}
        \item Batch Computing
        \item Dealing with Heterogeneous Environments (Clusters)
    \end{itemize}
\end{itemize}

\begin{itemize}
    \item Future Directions/Considerations
    \begin{itemize}
        \item Temporary Notebooks on Compute Nodes (users without accounts)
        \item Sharing Notebooks in Classroom Environments
        \item User Configurable Notebook Environments and Reproducible Research w/ Singularity
    \end{itemize}
\end{itemize}

\end{abstract}

\end{document}
